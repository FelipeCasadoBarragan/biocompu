\documentclass{article}

\usepackage[utf8]{inputenc}
\usepackage[spanish]{babel}
\usepackage{hyperref}
\usepackage{listings}

\title{Regresión}
\author{Fabián Villena y Felipe Arias}
\date{Julio 2023}

\begin{document}

\maketitle

Se le pide seleccionar el mejor modelo a la hora de predecir el índice de masa corporal de un paciente dados sus hábitos y condición física\footnote{\url{https://doi.org/10.1016/j.dib.2019.104344}}.

En particular, trabajaremos con un conjunto de datos que contiene información sobre hábitos alimenticios y condición física recopilados a través de una encuesta electrónica en países latinoamericanos. Usted deberá predecir sobre el atributo \texttt{BMI} (\textit{Body Mass Index})

Los resultados esperados son que puedan seleccionar la mejor combinación de hiperparámetros y modelos utilizando técnicas de validación de modelos. 

\begin{center}
    \url{https://github.com/fvillena/biocompu/blob/2023/data/obesity.csv}
\end{center}

\section*{Preguntas}

Responda las siguientes preguntas en un \textit{Jupyter Notebook} con código desarrollado en el lenguaje de programación Python.

\begin{enumerate}
	\item Verifique y preprocese si es necesario las características.
    \item Separe el conjunto de datos en un subconjunto de entrenamiento y prueba.
    \item Genere una grilla de hiperparámetros para el algoritmo \textit{Random Forests}.
    \item Optimice los hiperparámetros del algoritmo Random Forests sobre el subconjunto de entrenamiento utilizando la técnica de \textit{Grid-Search} y \textit{Cross-Validation} para validar los resultados.
    \item Prediga sobre el subconjunto de prueba utilizando un modelo ajustado con los mejores hiperparámetros.
    \item Calcule al menos dos métricas de rendimiento para las predicciones realizadas anteriormente.
\end{enumerate}

\end{document}

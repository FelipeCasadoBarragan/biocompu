\documentclass{article}

\usepackage[utf8]{inputenc}
\usepackage[spanish]{babel}
\usepackage{hyperref}
\usepackage{listings}

\title{Agrumapiento}
\author{Fabián Villena y Felipe Arias}
\date{Julio 2023}

\begin{document}

\maketitle

Se le pide detectar la existencia de grupos de objetos en un conjunto de datos de características de semillas.

En particular, trabajaremos con un conjunto de datos que contiene múltiples atributos físicos de semillas y que se cree que tiene 3 variedades, pero no se tienen las clases de cada una de las semillas.

Los resultados esperados son que apliquen una técnica de agrupamiento y sugieran el grupo al cual pertenece cada una de las semillas. 

\begin{center}
    \url{https://github.com/fvillena/biocompu/blob/2023/data/seeds.csv}
\end{center}

\section*{Preguntas}

Responda las siguientes preguntas en un \textit{Jupyter Notebook} con código desarrollado en el lenguaje de programación Python.

\begin{enumerate}
	\item Verifique y preprocese si es necesario las características.
    \item Realice agrupamiento con los algoritmos \textit{k-Means} y Agrupamiento Jerárquico utilizando distintas cantidades de grupos.
    \item Utilice el método del codo para seleccionar una buena cantidad de grupos.
    \item Explore los grupos sugeridos por los métodos ajustados con la cantidad de grupos encontrada por el método del codo.
\end{enumerate}

\end{document}

\documentclass{article}

\usepackage[utf8]{inputenc}
\usepackage[spanish]{babel}
\usepackage{hyperref}
\usepackage{listings}

\title{Clasificación}
\author{Fabián Villena y Felipe Arias}
\date{Julio 2023}

\begin{document}

\maketitle

Se le pide seleccionar el mejor modelo a la hora de predecir el tipo de medicamento adecuado para un paciente.

En particular, trabajaremos con un conjunto de datos que contiene información sobre la clasificación de medicamentos basada en la información general del paciente y su diagnóstico. La idea es construir modelos de aprendizaje automático para predecir el tipo de fármaco (\textit{Drug}) que podría ser adecuado para el paciente.

Los resultados esperados son que puedan realizar un flujo clásico al momento de crear modelos de aprendizaje supervisado. 

\begin{center}
    \url{https://github.com/fvillena/biocompu/blob/2023/data/drug.csv}
\end{center}

\section*{Preguntas}

Responda las siguientes preguntas en un \textit{Jupyter Notebook} con código desarrollado en el lenguaje de programación Python.

\begin{enumerate}
	\item Utilizar la técnica de \textit{one-hot encoding} para transformar variables categoricas a numéricas.
    \item Separe el conjunto de datos en un subconjunto de entrenamiento y prueba
    \item Entrene múltiples modelos de clasificación basados en Support Vector Machines con distintos hiperparámetros.
    \item Prediga sobre el subconjunto de prueba utilizando cada uno de los modelos ajustados.
    \item Calcule al menos dos métricas de rendimiento para cada uno de los modelos de clasificación.
    \item Seleccione la mejor combinación de hiperparámetros.
\end{enumerate}

\end{document}
